\section{Experiment and Results}

\subsection{Experiment Setup}

\subsubsection{Data Collection}

\subsubsection{TBC}

\subsubsection{Sentiment Analysis}

We use a lexicon called SentiWordNet for scoring the words in our corpus.
We filtered the tweets originated from only users themself, which means retweets will be removed.
The corpus is first preprocessed to remove elements mentioned in Tab.\ref{table:elementsRemoved}:

\begin{table}[h]
  \caption{Elements to be removed when proprocessing}
  \label{table:elementsRemoved}
  \centering
  \renewcommand{\tabularxcolumn}{m} % we want center vertical alignment
  \begin{tabularx}{\textwidth}{l  l || l  l}
    \toprule
    \textbf{Element} & \textbf{Examples} & \textbf{Element}    & \textbf{Examples}
    \tabularnewline \midrule
    URLs
                     &
    http://foo.bar   & Blank spaces      &
    \tabularnewline \hline
    Mentions to other users
                     & @Bot              & Single letter words & a b c
    \tabularnewline \hline
    Hashtags
                     & \#botRise         & Numbers             & 1994 233
    \tabularnewline \hline
    Twitter reserved words
                     & RT via            &
                     &
    \tabularnewline \bottomrule
  \end{tabularx}
\end{table}

We try not to remove punctuations and stopwords because we need to do Part-of-Speech(POS) tagging after tokenizing. When we remove numbers when try to remain the years (from 1900 to 2100).

Once the POS tags of words are generated. We up the words in SentiWordNet to determine the scoring for positiveness, negativeness and objectiveness. A score of one is generated

\subsubsection{Emotion Analysis}

\subsection{Results}
