\section{Experiment and Results}

\subsection{Experiment Setup}

\subsubsection{Data Collection}

\subsubsection{TBC}

\subsubsection{Sentiment \& Emotion Analysis}

Tweets are filtered and only tweets with texts originate from uses themselve remain, which means the likes and directly retweets are filtered.

The corpora are then preprocessed to remove elements mentioned in Table.\ref{table:elementsRemoved}:

\begin{table}[h]
  \caption{Elements to be removed when proprocessing}
  \label{table:elementsRemoved}
  \centering
  \renewcommand{\tabularxcolumn}{m} % we want center vertical alignment
  \begin{tabularx}{\textwidth}{l  l || l  l}
    \toprule
    \textbf{Element} & \textbf{Examples} & \textbf{Element}    & \textbf{Examples}
    \tabularnewline \midrule
    URLs
                     &
    http://foo.bar   & Blank spaces      &
    \tabularnewline \hline
    Mentions to other users
                     & @Bot              & Single letter words & a b c
    \tabularnewline \hline
    Hashtags
                     & \#botRise         & Numbers             & 1994 233
    \tabularnewline \hline
    Twitter reserved words
                     & RT via            &
                     &
    \tabularnewline \bottomrule
  \end{tabularx}
\end{table}

When we remove numbers we try to remain the years (from 1900 to 2100).
We try not to remove punctuations and stopwords because we need to do Part-of-Speech (POS) tagging after tokenizing. Both tokenizing and POS tagging is done by NLTK \citep{NLTK}.

We use lexicons to score the words in our corpus.
SentiWordNet is used for sentiment polarity analysis and NRC Word-Emotion Association Lexicon (EmoLex) \citep{Mohammad13} is for emotion analysis.

Once the POS tags of words are generated. We search the synonyms of words in SentiWordNet to determine the scoring for positiveness, negativeness and objectiveness by caculating means among synonyms. Meanwhile EmoLex is used to perform emotion analysis on 10 emotions. Scores of one tweet are generated caculating the means of the scores of all the words after preprocessing.

The result data applying SentiWordNet are shown in Table.\ref{table:sentiResult}, and result produced by using EmoLex are shown in Table.\ref{table:emolexResult}.

We also counted adjectives with top ?? frequencies in soldiers and civilians corpora, for we think that adjectives have more subjective meanings than verbs, nouns, etc. The list of adjectives are shown in Table.\ref{}.


\begin{table}[h]
  \caption{Results of sentiment analysis using SentiWordNet}
  \label{table:sentiResult}
  \centering
  \renewcommand{\tabularxcolumn}{m} % we want center vertical alignment
  \begin{tabularx}{\textwidth}{l l | l l l l l}
    \toprule
              &          & \textbf{Valid Cnt.} & \textbf{Valid Len.}    & \textbf{Possitive.}    & \textbf{Negative.}     & \textbf{Objective.}
    \tabularnewline \midrule
    Soldiers  & Mean
              & 3179.54* & 16.450              & 257.58$\times 10^{-4}$ & 196.10$\times 10^{-4}$ & 3371.7$\times 10^{-4}$
    \tabularnewline
    n=208     & Std.
              & 5041.70  & 6.6427              & 78.586$\times 10^{-4}$ & 64.386$\times 10^{-4}$ & 750.49$\times 10^{-4}$
    \tabularnewline \hline \hline
    Civilians & Mean
              & 2143.66* & 14.293              & 262.65$\times 10^{-4}$ & 177.39$\times 10^{-4}$ & 3530.5$\times 10^{-4}$
    \tabularnewline
    n=301     & Std.
              & 5286.12  & 5.2067              & 87.432$\times 10^{-4}$ & 64.786$\times 10^{-4}$ & 720.09$\times 10^{-4}$
    \tabularnewline \bottomrule
  \end{tabularx}
\end{table}


\begin{table}[h]
  \caption{Results of emotion analysis using EmoLex}
  \label{table:emolexResult}
  \centering
  \renewcommand{\tabularxcolumn}{m} % we want center vertical alignment
  \begin{tabularx}{\textwidth}{l l l l l l}
    \toprule
    Soldiers: n=208
    \tabularnewline \midrule
     & \textbf{Trust}   & \textbf{Anger} & \textbf{Surprise}      & \textbf{Joy}           & \textbf{Positive.}
    \tabularnewline \midrule
    Mean
     & 3179.54          & 16.450         & 257.58$\times 10^{-4}$ & 196.10$\times 10^{-4}$ & 3371.7$\times 10^{-4}$
    \tabularnewline
    Std.
     & 5041.70          & 6.6427         & 78.586$\times 10^{-4}$ & 64.386$\times 10^{-4}$ & 750.49$\times 10^{-4}$
    \tabularnewline \midrule
     & \textbf{Disgust} & \textbf{Fear}  & \textbf{Anticipat.}    & \textbf{Sadness}       & \textbf{Negative.}
    \tabularnewline \midrule
    Mean
     & 2143.66          & 14.293         & 262.65$\times 10^{-4}$ & 177.39$\times 10^{-4}$ & 3530.5$\times 10^{-4}$
    \tabularnewline
    Std.
     & 5286.12          & 5.2067         & 87.432$\times 10^{-4}$ & 64.786$\times 10^{-4}$ & 720.09$\times 10^{-4}$
    \tabularnewline \hline \hline
    Civilians: n=301
    \tabularnewline \midrule
     & \textbf{Trust}   & \textbf{Anger} & \textbf{Surprise}      & \textbf{Joy}           & \textbf{Positive.}
    \tabularnewline \midrule
    Mean
     & 3179.54          & 16.450         & 257.58$\times 10^{-4}$ & 196.10$\times 10^{-4}$ & 3371.7$\times 10^{-4}$
    \tabularnewline
    Std.
     & 5041.70          & 6.6427         & 78.586$\times 10^{-4}$ & 64.386$\times 10^{-4}$ & 750.49$\times 10^{-4}$
    \tabularnewline \midrule
     & \textbf{Disgust} & \textbf{Fear}  & \textbf{Anticipat.}    & \textbf{Sadness}       & \textbf{Negative.}
    \tabularnewline \midrule
    Mean
     & 2143.66          & 14.293         & 262.65$\times 10^{-4}$ & 177.39$\times 10^{-4}$ & 3530.5$\times 10^{-4}$
    \tabularnewline
    Std.
     & 5286.12          & 5.2067         & 87.432$\times 10^{-4}$ & 64.786$\times 10^{-4}$ & 720.09$\times 10^{-4}$
    \tabularnewline \bottomrule
  \end{tabularx}
\end{table}
