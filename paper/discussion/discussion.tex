\section{Discussion}

The entire idea for our research was to check the mental condition of veterans to that of to a normal civilian and eventually compare the sentiment traits. Our research involved 488 people that included the civilians as well as the war veterans, shows that there is definitely an effect of war that can be reflected by the word selection and other traits they use while tweeting. One of the analysis shows that the number of "curses" used in by the war veterans is much greater than that used by ordinary civilians. The word count for curse word by war veterans accounts to be 28,249 which is much greater than that curse world count used by civilians which is at 8983. On an average, Veterans employed profanity 3 times more than a civilian did.

From Table \ref{table:sentiResult} we cannot get much inference on the sentiment part, instead we find that soldiers are more likely to post lengthy tweets (see the numbers with *).

We can see from Table \ref{table:emolexResult} that corpus of soldiers' tweets has more \enquote{negative} emotions like Disgust, Fear, Anger and Sadness. The corpus of soldiers' tweets is judged as negative on the whole. While civilians' corpus tends to be more positive, with better metrics on Surprise, Anticipation, and Joy. One interesting emotion is Trust, from Plutchik's wheel of emotions \ref{fig:wheel} we can infer that Trust is a kind of emotion related to submission, acceptance and admiration, which is related to soldiers' loyalty obeying the commands. While Surprise is related to disapproval and distraction, which can somewhat indicate the quality of disorder among internet users.

With the list of adjectives (Table \ref{table:wordsComparison}) we can see that soldiers are more involved in political topics. It might be that soldiers/veterans are more involved in political events fighting for their rights, also they usually have closer relations to governments and military.
