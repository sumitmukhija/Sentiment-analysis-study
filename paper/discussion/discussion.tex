\section{Discussion}

From Table.\ref{table:sentiResult} we cannot get much inference on the sentiment part, instead we find that soldiers are more likely to send long texts (see the numbers with *).

We can see from Table.\ref{table:emolexResult} that corpus of soldiers' tweets has more \enquote{negative} emotions like Disgust, Fear, Anger and Sadness. The corpus of soldiers' tweets is judged as negative on the whole. While civilians' corpus tend to be more possitive, with better metrics on Surprise, Anticipation, and Joy. One interesting item is Trust, from Plutchik’s wheel of emotions \ref{} we can infer that Trust is a kind of emotion related to submission, acceptance and admiration, which is related to soldiers' loyalty obeying the commands. While Surprise is related to disaproval and distraction, which can somewhat indicating the quality of disorder among internet users.

With the list of adjectives (Table.\ref{table:wordsComparison}) we can see that soldiers are more involved in political topics.
