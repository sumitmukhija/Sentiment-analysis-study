\section{Discussion}

The entire idea for our research was to check the mental condition of veterans or any army person to that of to a normal civilian and therefore draw a sentimental analysis. Our research done on nearly 488 people that include the civilian’s as well as the vets shows that there is definitely effect of wars that can be reflected from the type of words they use while tweeting. One of the analysis shows that the number of "curses" used in by the vets are much more greater that that used by an ordinary civilians. The word count for curse word by vets accounts to be 28,249 which is much greater than that curse world count used by civilians which is at 8,983.

From Table.\ref{table:sentiResult} we cannot get much inference on the sentiment part, instead we find that soldiers are more likely to send long texts (see the numbers with *).

We can see from Table.\ref{table:emolexResult} that corpus of soldiers' tweets has more \enquote{negative} emotions like Disgust, Fear, Anger and Sadness. The corpus of soldiers' tweets is judged as negative on the whole. While civilians' corpus tend to be more possitive, with better metrics on Surprise, Anticipation, and Joy. One interesting item is Trust, from Plutchik’s wheel of emotions \ref{fig:wheel} we can infer that Trust is a kind of emotion related to submission, acceptance and admiration, which is related to soldiers' loyalty obeying the commands. While Surprise is related to disaproval and distraction, which can somewhat indicating the quality of disorder among internet users.

With the list of adjectives (Table.\ref{table:wordsComparison}) we can see that soldiers are more involved in political topics. It might be that soldiers / veterans are more involved in political events fighting for their rights, also they usually have closer relations to governments and military.
