\documentclass[english,a4paper,11pt]{article}

\usepackage[utf8]{inputenc}
\usepackage{natbib}
\usepackage{graphicx}  %%% for including graphics
\graphicspath{{./images/}}
\usepackage{url}       %%% for including URLs
\usepackage{times}
\usepackage[margin=25mm]{geometry}

%%% custom packages
\usepackage{booktabs}
\usepackage{tabularx}
\usepackage{hyperref}
\usepackage[autostyle=true]{csquotes}
\usepackage{xpatch}
%%% custom packages

\title{Sentiment Analysis of Soldiers' Tweets - Comparison with civilians (TBC)}
\date{\today}

\author{
  Sumit Mukhija, Rachit Rastogi,\\
  Chao Chen, Chen Wang, Chetan Prasad\\
  School of Computer Science and Statistics, Trinity College Dublin\\
  \texttt{\{mukhijas, rrastogi, chenc1, wangc5, cprasad\}@tcd.ie}
}

\begin{document}
\maketitle
\thispagestyle{empty}
\pagestyle{empty}

\begin{abstract}
  The concern to veterans' mental health should be made. Existing works show that
  mental health changes caused by wars can be reflected in linguistic features of
  the social media texts. In order to detect and compare those changes we collected
  data from 20 soldiers' tweets and examined them with a list of positive and negative
  adjectives to identify the polarity and do a comparison with normal users'
  tweets. The total counts of tweets vary from 57 to 39,000. We identified the
  difference between normal users and soldiers and we did a close look to the
  result with discussion. \\
  \textbf{Keywords:} Twitter, tweet, sentiment, emotion, soldier, SentiWordNet, EmoLex, lexicon
\end{abstract}

\section{Introduction}

Social media platforms and microblogging websites are some of the most popular online stages for people to express their views. Twitter undeniably is one of the leading applications in this assortment. People use Twitter to post their real-time opinions in the form of tweets. These tweets can be analyzed and certain inferences can be extracted. These inferences can subsequently be used for academic and business purposes.

One of the primary reasons that make Twitter a feasible choice is the diverse nature of users. In this research, we intend to analyze and compare the tweets of war-veterans and the general public. We believe wars have an impact on soldiers' psychological and emotional states. We try to prove this hypothesis by comparing their tweets to the tweets posted by the civilians. We collect public data using Twitter API of both veterans/soldiers and civilians. The accounts selected for the research all meet the criteria of having a minimum of 50 tweets and have had the account for longer than a year and do not belong to any organisation. The tweets collected is processed using SentiWordNet to recognize the polarity of the tweets, we also count the number of words, the number of cuss words used. The data extracted from the processing of veteran/soldier tweets are compared with those of the civilians.

The remainder of the paper is organized as follows. We examine on the literature related to the topic, with papers related to previous works on the mental health of veterans, available databases on sentiment analysis and previous works done on sentiment analysis on social media in Section 2. In Section 3, we introduce our dataset and the experiment done on the dataset, with the results we have. In Section 4, we have a deep look into the result and bring the discussion. In Section 5 and 6, we conclude and bring up future works needed for the topic.

\section{Background}

\subsection{Previous Work on Mental Health of Veterans}

In order to make a medical diagnosis for patients, psychologists often use the
linguistic content and expression of patients to judge their emotional changes
and mental state according to previous research in psychology and linguistic.
The clinical diagnosis efficiency has been greatly improved because of the
progress of science and technology, especially in computational linguistics.
In addition, the widespread of social media such as Facebook, Twitter and
Instagram, has provided mental researchers with a large scale of data.
Therefore, they could easily use the collected dataset and machine learning
techniques for sentiment analysis. Linguistic contents which users posted on
social media have been proved to be the basis for evaluating a person's mental
state \citep{becauseIwastoldsomuch} \citep{GUNTUKU201743}. However,
the majority of research targets are normal people.

In this paper, veterans will be regarded as research targets. Westgate in
\citep{doi:10.1176/appi.ps.201400283} has come up with a method about the evaluation
of veterans' suicide risk. This paper will concentrate on analyzing
the impact of the war on veterans' mental state through the Twitters posted by
themselves before and after the war instead of focusing on the prediction of
the suicide risk of veterans. In addition, the comparison with the twitters
released by ordinary users will be presented. Finally, comprehensive
sentiment analysis of veterans will be summarized.

\subsection{Available Database on Sentiment Analysis}

Sentiment analysis is a very useful method for analysing the sentiment of an
article, tweets or reviews. The advances in Natural Language processing and
linguistic research have led to the development of different methods of
sentiment analysis. The sentiment analysis is an integral part of our work and
it is the fulcrum on which our hypothesis hangs on. Determining not just the
sentiment of a text but even the topic \citep{10.1145/1645953.1646003} on which
it is written on is one of the interesting works in this field. Another work
makes use of the lexicons in the text and word dictionaries to extract the
sentiment behind it \citep{10.5555/2002472.2002491}. There is another work that
trains a model to learn not just the words but to also capture the sentiment
behind it \citep{taboada2011lexicon}.

In this work, we do sentiment analysis on tweets and we compare the tweets of
two types of users. We utilise a simple model which basically just classifies
the tweets into Happy and sad/depressing sentiments. This is sufficient
considering our work is to identify the difference in the mental states between
the soldiers who have served in wars and common people.

\subsection{Sentiment Analysis on Social Media}

Emotion and sentiment are treated as different concepts in psychology. The
definition of \enquote{emotion} is a complex psychological state, which plays an
important role in operating motivators. For \enquote{sentiment}, it is created based on
emotion to refer to a mental attitude. A survey providing more information can
be referred to by \citep{yue2018survey}.

In sentiment analysis, the typical task is finding the polarities of the given
texts. The tests are probably positive, negative or neutral. The approach is
often counting the word using and produce scores due to the lexicons.

There are commonly two approaches - analysing users' social activities and calculate linguistic features of user-generated texts. The sentiment analysis mainly focuses on short texts(tweets) generated from Twitter accounts, since most of the data is public by default and easy to obtain online. Also, the tweets are short(limited to 280 characters) and often appears with spelling mistakes and slangs. A tweet often comes with other features like spreading tweets(retweet) from other accounts. These methods mentioned above make the analysis on tweets a paradigm to explore.

\section{Experiment and Results}

\subsection{Experiment Setup}

\subsubsection{Data Collection}

\subsubsection{TBC}

\subsubsection{Sentiment \& Emotion Analysis}

Tweets are filtered and only tweets with texts originate from uses themselve remain, which means the likes and directly retweets are filtered.

The corpora are then preprocessed to remove elements mentioned in Table.\ref{table:elementsRemoved}:

\begin{table}[h]
  \caption{Elements to be removed when proprocessing}
  \label{table:elementsRemoved}
  \centering
  \renewcommand{\tabularxcolumn}{m} % we want center vertical alignment
  \begin{tabularx}{\textwidth}{l  l || l  l}
    \toprule
    \textbf{Element} & \textbf{Examples} & \textbf{Element}    & \textbf{Examples}
    \tabularnewline \midrule
    URLs
                     &
    http://foo.bar   & Blank spaces      &
    \tabularnewline \hline
    Mentions to other users
                     & @Bot              & Single letter words & a b c
    \tabularnewline \hline
    Hashtags
                     & \#botRise         & Numbers             & 1994 233
    \tabularnewline \hline
    Twitter reserved words
                     & RT via            &
                     &
    \tabularnewline \bottomrule
  \end{tabularx}
\end{table}

When we remove numbers we try to remain the years (from 1900 to 2100).
We try not to remove punctuations and stopwords because we need to do Part-of-Speech (POS) tagging after tokenizing. Both tokenizing and POS tagging is done by NLTK \citep{NLTK}.

We use lexicons to score the words in our corpus.
SentiWordNet is used for sentiment polarity analysis and NRC Word-Emotion Association Lexicon (EmoLex) \citep{Mohammad13} is for emotion analysis.

Once the POS tags of words are generated. We search the synonyms of words in SentiWordNet to determine the scoring for positiveness, negativeness and objectiveness by caculating means among synonyms. Meanwhile EmoLex is used to perform emotion analysis on 10 emotions. Scores of one tweet are generated caculating the means of the scores of all the words after preprocessing.

The result data applying SentiWordNet are shown in Table.\ref{table:sentiResult}, and result produced by using EmoLex are shown in Table.\ref{table:emolexResult}.

We also counted adjectives with top 100 frequencies in soldiers and civilians corpora, for we think that adjectives have more subjective meanings than verbs, nouns, etc. We discovered some words with more "political" meanings appear to be different in the lists of two corpora. The list of adjectives are shown in Table.\ref{table:wordsComparison}.


\begin{table}[h]
  \caption{Results of sentiment analysis using SentiWordNet}
  \label{table:sentiResult}
  \centering
  \renewcommand{\tabularxcolumn}{m} % we want center vertical alignment
  \begin{tabularx}{\textwidth}{l l | l l l l l}
    \toprule
              &          & \textbf{Valid Cnt.} & \textbf{Valid Len.}    & \textbf{Possitive.}    & \textbf{Negative.}     & \textbf{Objective.}
    \tabularnewline \midrule
    Soldiers  & Mean
              & 3179.54* & 16.450              & 257.58$\times 10^{-4}$ & 196.10$\times 10^{-4}$ & 3371.7$\times 10^{-4}$
    \tabularnewline
    n=208     & Std.
              & 5041.70  & 6.6427              & 78.586$\times 10^{-4}$ & 64.386$\times 10^{-4}$ & 750.49$\times 10^{-4}$
    \tabularnewline \hline \hline
    Civilians & Mean
              & 2143.66* & 14.293              & 262.65$\times 10^{-4}$ & 177.39$\times 10^{-4}$ & 3530.5$\times 10^{-4}$
    \tabularnewline
    n=280     & Std.
              & 5286.12  & 5.2067              & 87.432$\times 10^{-4}$ & 64.786$\times 10^{-4}$ & 720.09$\times 10^{-4}$
    \tabularnewline \bottomrule
  \end{tabularx}
\end{table}


\begin{table}[h]
  \caption{Results of emotion analysis using EmoLex}
  \label{table:emolexResult}
  \centering
  \renewcommand{\tabularxcolumn}{m} % we want center vertical alignment
  \begin{tabularx}{\textwidth}{l l l l l l}
    \toprule
    Soldiers: n=208
    \tabularnewline \midrule
     & \textbf{Trust}+   & \textbf{Anger}+ & \textbf{Surprise}    & \textbf{Joy}      & \textbf{Positive.}
    \tabularnewline \midrule
    Mean $\times 10^{-4}$
     & 422.84            & 167.17          & 149.99               & 312.51            & 637.50
    \tabularnewline
    Std. $\times 10^{-4}$
     & 154.94            & 83.143          & 62.113               & 151.64            & 213.55
    \tabularnewline \midrule
     & \textbf{Disgust}+ & \textbf{Fear}+  & \textbf{Anticipat.}  & \textbf{Sadness}+ & \textbf{Negative.}+
    \tabularnewline \midrule
    Mean$\times 10^{-4}$
     & 122.09            & 193.43          & 295.57               & 149.34            & 339.61
    \tabularnewline
    Std.$\times 10^{-4}$
     & 74.795            & 89.380          & 112.40               & 66.053            & 148.79
    \tabularnewline \hline \hline
    Civilians: n=280
    \tabularnewline \midrule
     & \textbf{Trust}    & \textbf{Anger}  & \textbf{Surprise}+   & \textbf{Joy}+     & \textbf{Positive.}+
    \tabularnewline \midrule
    Mean $\times 10^{-4}$
     & 399.72            & 132.03          & 163.72               & 349.44            & 650.63
    \tabularnewline
    Std.$\times 10^{-4}$
     & 170.45            & 80.189          & 108.13               & 224.58            & 269.03
    \tabularnewline \midrule
     & \textbf{Disgust}  & \textbf{Fear}   & \textbf{Anticipat.}+ & \textbf{Sadness}  & \textbf{Negative.}
    \tabularnewline \midrule
    Mean$\times 10^{-4}$
     & 98.934            & 163.78          & 330.09               & 131.82            & 283.00
    \tabularnewline
    Std.$\times 10^{-4}$
     & 82.884            & 108.23          & 152.86               & 87.180            & 160.08
    \tabularnewline \bottomrule
  \end{tabularx}
\end{table}


\begin{table}[h]
  \caption{List of word rankings and frequencies}
  \label{table:wordsComparison}
  \centering
  \renewcommand{\tabularxcolumn}{m} % we want center vertical alignment
  \begin{tabularx}{\textwidth}{l | l | l || l | l | l}
    \toprule
    \textbf{Word} & \textbf{Soldiers} & \textbf{Civilians} & \textbf{Word} & \textbf{Soldiers} & \textbf{Civilians}
    \tabularnewline \midrule
    military
                  &
    4861 (17th)   & dead              &
    \tabularnewline \hline
    american
                  & 4193 (24th)       & human              & a b c
    \tabularnewline \hline
    political
                  & 2305 (40th)       & local              & 1994 233
    \tabularnewline \hline
    medical
                  & 1914 (47th)       & democratic
                  &
    \tabularnewline \hline
    public
                  & RT via            & illegal
                  &
    \tabularnewline \hline
    social
                  & RT via            & foreign
                  &
    \tabularnewline \hline
    sick
                  & RT via            & poor
                  &
    \tabularnewline \hline
    personal
                  & RT via            & republican
                  &
    \tabularnewline \bottomrule
  \end{tabularx}
\end{table}

\section{Discussion}

The entire idea for our research was to check the mental condition of veterans to that of to a normal civilian and eventually compare the sentiment traits. Our research involved 488 people that included the civilians as well as the war veterans, shows that there is definitely an effect of war that can be reflected by the word selection and other traits they use while tweeting. One of the analysis shows that the number of "curses" used in by the war veterans is much greater than that used by ordinary civilians. The word count for curse word by war veterans accounts to be 28,249 which is much greater than that curse world count used by civilians which is at 8983. On an average, Veterans employed profanity 3 times more than a civilian did.

From Table \ref{table:sentiResult} we cannot get much inference on the sentiment part, instead we find that soldiers are more likely to post lengthy tweets (see the numbers with *).

We can see from Table \ref{table:emolexResult} that corpus of soldiers' tweets has more \enquote{negative} emotions like Disgust, Fear, Anger and Sadness. The corpus of soldiers' tweets is judged as negative on the whole. While civilians' corpus tends to be more positive, with better metrics on Surprise, Anticipation, and Joy. One interesting emotion is Trust, from Plutchik's wheel of emotions \ref{fig:wheel} we can infer that Trust is a kind of emotion related to submission, acceptance and admiration, which is related to soldiers' loyalty obeying the commands. While Surprise is related to disapproval and distraction, which can somewhat indicate the quality of disorder among internet users.

With the list of adjectives (Table \ref{table:wordsComparison}) we can see that soldiers are more involved in political topics. It might be that soldiers/veterans are more involved in political events fighting for their rights, also they usually have closer relations to governments and military.

\section{Conclusion}

\section{Future Works}

Thought we did try to answer our research question by applying two different analysis, there is always scope of improvement. One area where we like to work further is to extract data depending upon the timestamp. We wanted to get extract data depending on times when the veterans returned from a war and then compare it with their own tweets before the war. For the sentiment and emotion analysis, negators (e.g. not, isn't, won't) need to be considered for that they contribute strong and opposite effects towards polarity and word meaning. A cross comparison should be made as the work of \cite{emoIntenT} but on our corpora. These works can be done in future, and definitely worth exploring.


\clearpage

\bibliographystyle{chicago}
\bibliography{ref}

\end{document}
