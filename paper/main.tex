\documentclass[english,a4paper,11pt]{article}

\usepackage[utf8]{inputenc}
\usepackage{natbib}
\usepackage{graphicx}  %%% for including graphics
\graphicspath{{./images/}}
\usepackage{url}       %%% for including URLs
\usepackage{times}
\usepackage[margin=25mm]{geometry}

%%% custom packages
\usepackage{booktabs}
\usepackage{tabularx}
\usepackage{hyperref}
\usepackage[autostyle=true]{csquotes}
\usepackage{xpatch}
%%% custom packages

\title{Lexical and Non-Lexical Analysis of Soldiers' Tweets\\- With Sentiment and Emotion Metrics}
\date{\today}

\author{
  Chao Chen, Chetan Prasad, Chen Wang, \\
  Rachit Rastogi, Sumit Mukhija \\
  School of Computer Science and Statistics, Trinity College Dublin\\
  \texttt{\{chenc1, cprasad, wangc5, rrastogi, mukhijas\}@tcd.ie}
}

\begin{document}
\maketitle
\thispagestyle{empty}
\pagestyle{empty}

\begin{abstract}
  A large number of war veterans are diagnosed with post traumatic stress disorder every day. This warrants a concern to evaluate the mental health of the war veterans. Existing works show that mental health changes caused by wars can be reflected in lexical and non-lexical features of the social media texts. In order to detect and compare those changes we collected data from 208 soldiers' tweets (n=6,61,342) with 280 civilians' tweets (n=6,00,173). We examined them with lexical and non-lexical aspects. Elements and attributes of tweets are processed and included as non-lexical features. Sentiment and emotion analysis is made on the corpora. We had a close look at the results with discussion and identified the differences between civilians and soldiers in both positive and negative directions. \\
  \textbf{Keywords:} Twitter, tweet, soldier, veteran, lexical, SentiWordNet, EmoLex, lexicon, sentiment, emotion
\end{abstract}

\input{introduction/introduction.tex}
\section{Background}

\subsection{Previous Work on Mental Health of Veterans}

In order to make a medical diagnosis for patients, psychologists often use the
linguistic content and expression of patients to judge their emotional changes
and mental state according to previous research in psychology and linguistic.
The clinical diagnosis efficiency has been greatly improved because of the
progress of science and technology, especially in computational linguistics.
In addition, the widespread of social media such as Facebook, Twitter and
Instagram, has provided mental researchers with a large scale of data.
Therefore, they could easily use the collected dataset and machine learning
techniques for sentiment analysis. Linguistic contents which users posted on
social media have been proved to be the basis for evaluating a person's mental
state \citep{becauseIwastoldsomuch} \citep{GUNTUKU201743}. However,
the majority of research targets are normal people.

In this paper, veterans will be regarded as research targets. Westgate in
\citep{doi:10.1176/appi.ps.201400283} has come up with a method about the evaluation
of veterans' suicide risk. This paper will concentrate on analyzing
the impact of the war on veterans' mental state through the Twitters posted by
themselves before and after the war instead of focusing on the prediction of
the suicide risk of veterans. In addition, the comparison with the twitters
released by ordinary users will be presented. Finally, comprehensive
sentiment analysis of veterans will be summarized.

\subsection{Available Database on Sentiment Analysis}

Sentiment analysis is a very useful method for analysing the sentiment of an
article, tweets or reviews. The advances in Natural Language processing and
linguistic research have led to the development of different methods of
sentiment analysis. The sentiment analysis is an integral part of our work and
it is the fulcrum on which our hypothesis hangs on. Determining not just the
sentiment of a text but even the topic \citep{10.1145/1645953.1646003} on which
it is written on is one of the interesting works in this field. Another work
makes use of the lexicons in the text and word dictionaries to extract the
sentiment behind it \citep{10.5555/2002472.2002491}. There is another work that
trains a model to learn not just the words but to also capture the sentiment
behind it \citep{taboada2011lexicon}.

In this work, we do sentiment analysis on tweets and we compare the tweets of
two types of users. We utilise a simple model which basically just classifies
the tweets into Happy and sad/depressing sentiments. This is sufficient
considering our work is to identify the difference in the mental states between
the soldiers who have served in wars and common people.

\subsection{Sentiment Analysis on Social Media}

Emotion and sentiment are treated as different concepts in psychology. The
definition of \enquote{emotion} is a complex psychological state, which plays an
important role in operating motivators. For \enquote{sentiment}, it is created based on
emotion to refer to a mental attitude. A survey providing more information can
be referred to by \citep{yue2018survey}.

In sentiment analysis, the typical task is finding the polarities of the given
texts. The tests are probably positive, negative or neutral. The approach is
often counting the word using and produce scores due to the lexicons.

There are commonly two approaches - analysing users' social activities and calculate linguistic features of user-generated texts. The sentiment analysis mainly focuses on short texts(tweets) generated from Twitter accounts, since most of the data is public by default and easy to obtain online. Also, the tweets are short(limited to 280 characters) and often appears with spelling mistakes and slangs. A tweet often comes with other features like spreading tweets(retweet) from other accounts. These methods mentioned above make the analysis on tweets a paradigm to explore.

\section{Experiment and Results}

\subsection{Experiment Setup}

\subsubsection{Data Collection}

\subsubsection{TBC}

\subsubsection{Sentiment Analysis}

We use a lexicon called SentiWordNet to

\subsubsection{Emotion Analysis}

\subsection{Results}

\section{Discussion}

The entire idea for our research was to check the mental condition of veterans or any army person to that of to a normal civilian and therefore draw a sentimental analysis. Our research done on nearly 488 people that include the civilian’s as well as the vets shows that there is definitely effect of wars that can be reflected from the type of words they use while tweeting. One of the analysis shows that the number of "curses" used in by the vets are much more greater that that used by an ordinary civilians. The word count for curse word by vets accounts to be 28,249 which is much greater than that curse world count used by civilians which is at 8,983.

From Table.\ref{table:sentiResult} we cannot get much inference on the sentiment part, instead we find that soldiers are more likely to send long texts (see the numbers with *).

We can see from Table.\ref{table:emolexResult} that corpus of soldiers' tweets has more \enquote{negative} emotions like Disgust, Fear, Anger and Sadness. The corpus of soldiers' tweets is judged as negative on the whole. While civilians' corpus tend to be more possitive, with better metrics on Surprise, Anticipation, and Joy. One interesting item is Trust, from Plutchik’s wheel of emotions \ref{fig:wheel} we can infer that Trust is a kind of emotion related to submission, acceptance and admiration, which is related to soldiers' loyalty obeying the commands. While Surprise is related to disaproval and distraction, which can somewhat indicating the quality of disorder among internet users.

With the list of adjectives (Table.\ref{table:wordsComparison}) we can see that soldiers are more involved in political topics. It might be that soldiers / veterans are more involved in political events fighting for their rights, also they usually have closer relations to governments and military.

\section{Conclusion}

The results that we received after extracting and processing the data for the soldiers and the civilians conclude that there exists a sense of emotional dilemma to the veterans when compared to normal people. They do show characteristics of using a higher amount of curse words along with longer messages just to name a few. With the high amount of data that we had along with the preprocessing and methodologies used we are certain that we were able to prove our research area and come to a conclusion at the end of our research. Nevertheless, we feel that there is still a lot of work that exits in these areas and we hope that our findings enlighten other researchers to deep dive furthermore.

\section{Future Works}

Thought we did try to answer our research question by applying two different analysis, there is always scope of improvement. One area where we like to work further is to extract data depending upon the timestamp. We wanted to get extract data depending on times when the veterans returned from a war and then compare it with their own tweets before the war. For the sentiment and emotion analysis, negators (e.g. not, isn't, won't) need to be considered for that they contribute strong and opposite effects towards polarity and word meaning. A cross comparison should be made as the work of \cite{emoIntenT} but on our corpora. These works can be done in future, and definitely worth exploring.


\clearpage

\bibliographystyle{chicago}
\bibliography{ref}

\end{document}
