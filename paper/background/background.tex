\section{Background}

\subsection{Previous Work on Mental Health of Veterans}

In order to make a medical diagnosis for patients, psychologists often use the
linguistic content and expression of patients to judge their emotional changes
and mental state according to previous research in psychology and linguistic.
The clinical diagnosis efficiency has been greatly improved because of the
progress of science and technology, especially in computational linguistics.
In addition, the widespread of social media such as Facebook, Twitter and
Instagram, has provided mental researchers with a large scale of data.
Therefore, they could easily use the collected dataset and machine learning
techniques for sentiment analysis. Linguistic contents which users posted on
social media have been proved to be the basis for evaluating a person's mental
state \citep{becauseIwastoldsomuch} \citep{GUNTUKU201743}. However,
the majority of research targets are normal people.

In this paper, veterans will be regarded as research targets. Westgate in
\citep{doi:10.1176/appi.ps.201400283} has come up with a method about the evaluation
of veterans' suicide risk. This paper will concentrate on analyzing
the impact of the war on veterans' mental state through the Twitters posted by
themselves before and after the war instead of focusing on the prediction of
the suicide risk of veterans. In addition, the comparison with the twitters
released by ordinary users will be presented. Finally, comprehensive
sentiment analysis of veterans will be summarized.

\subsection{Available Database on Sentiment Analysis}

Sentiment analysis is a very useful method for analysing the sentiment of an
article, tweets or reviews. The advances in Natural Language processing and
linguistic research have led to the development of different methods of
sentiment analysis. The sentiment analysis is an integral part of our work and
it is the fulcrum on which our hypothesis hangs on. Determining not just the
sentiment of a text but even the topic \citep{10.1145/1645953.1646003} on which
it is written on is one of the interesting works in this field. Another work
makes use of the lexicons in the text and word dictionaries to extract the
sentiment behind it \citep{10.5555/2002472.2002491}. There is another work that
trains a model to learn not just the words but to also capture the sentiment
behind it \citep{taboada2011lexicon}.

In this work, we do sentiment analysis on tweets and we compare the tweets of
two types of users. We utilise a simple model which basically just classifies
the tweets into Happy and sad/depressing sentiments. This is sufficient
considering our work is to identify the difference in the mental states between
the soldiers who have served in wars and common people.

\subsection{Sentiment Analysis on Social Media}

Emotion and sentiment are treated as different concepts in psychology. The
definition of \enquote{emotion} is a complex psychological state, which plays an
important role in operating motivators. For \enquote{sentiment}, it is created based on
emotion to refer to a mental attitude. A survey providing more information can
be referred to by \citep{yue2018survey}.

In sentiment analysis, the typical task is finding the polarities of the given
texts. The tests are probably positive, negative or neutral. The approach is
often counting the word using and produce scores due to the lexicons.

The combination of machine learning and data mining techniques are the key part
of sentiment analysis on social media. There are commonly two approaches -
analysing users' social activities and calculate linguistic features of
user-generated texts. The sentiment analysis mainly focuses on short texts(tweets)
generated from Twitter accounts, since most of the data is public by default and
easy to obtain online. Also, the tweets are short(limited to 280 characters) and
often appears with spelling mistakes and slangs. A tweet often comes with other
features like spreading tweets(retweet) from other accounts. These mentioned
above makes the analysis on tweets a paradigm to explore.

\cite{robustTwitter2010Luciano} produced an approach to automatically analyse
sentiment on tweets with metainformation as retweets, hashtags, replies,
punctuations and emoticons. They also use sources of noisy labels in their
training data to test the robustness of the model.

\cite{agarwal2011sentiment} performed sentiment analysis on Twitter data using POS-
specific prior polarity features, the lexicon features and also the meta
features. They use a tree structure to represent tweets and used a partial tree
kernel to measure the similarity between two set of tweets.

\cite{opinionMiningLiang2013} developed a Unigram Naive Bayes model for
sentiment analysis on Twitter data. The $\chi^2$ feature extraction method
and Mutual Information were used to delete unwanted features, and then predictions
were made on the tweets whether they were positive or negative. They found that
the combination of prior polarity of words and their POS tags affects the most.

\cite{bhavsar2019sentiment} used a dataset from Kaggle and classify the people
emotions based on positive and negative reviews. The model they produced can
perform well on a large dataset.

The International Workshop on Semantic Evaluation(also known as SemEval)
started holding tasks related to sentiment analysis on social media since 2013.
The rankings of each tasks are made public also with solutions. A list of task
can be found in Table.\ref{table:SemEvalTasks}. % TODO task table

% TODO chage the table, kept as a example on table

\begin{table}[h]
  \caption{The tasks of SemEval on sentiment analysis of tweets}
  \label{table:SemEvalTasks}
  \centering
  \renewcommand{\tabularxcolumn}{m} % we want center vertical alignment
  \begin{tabularx}{\textwidth}{l >{\raggedright}X}
    \toprule
    \textbf{Workshop} & \textbf{Task}
    \tabularnewline \midrule
    SemEval-2013
                      &
    Task 2: Sentiment Analysis in Twitter \citep{SemEval2013Task2}
    \tabularnewline \hline
    SemEval-2014
                      &
    Task 9: Sentiment Analysis in Twitter \citep{SemEval2014Task9}
    \tabularnewline \hline
    SemEval-2015
                      &
    Task 10: Sentiment Analysis in Twitter \citep{SemEval2015Task10} \\
    Task 11: Sentiment Analysis of Figurative Language in Twitter
    \citep{SemEval2015Task11}
    \tabularnewline \hline
    SemEval-2016
                      &
    Task 4: Sentiment Analysis in Twitter \citep{SemEval2016Task4}   \\
    Task 6: Detecting Stance in Tweets \citep{SemEval2016Task6}
    \tabularnewline \hline
    SemEval-2017
                      &
    Task 4: Sentiment Analysis in Twitter \citep{SemEval2017Task4}   \\
    Task 6: \#HashtagWars: Learning a Sense of Humor \citep{SemEval2017Task6}
    \tabularnewline \hline
    SemEval-2018
                      &
    Task 1: Affect in Tweets \citep{SemEval2018Task1}                \\
    Task 3: Irony Detection in English Tweets \citep{SemEval2018Task3}
    \tabularnewline \hline
    SemEval-2019
                      &
    Task 5: HatEval: Multilingual Detection of Hate Speech Against Immigrants
    and Women in Twitter \citep{SemEval2019Task5}                    \\
    Task 6: OffensEval: Identifying and Categorizing Offensive Language in
    Social Media \citep{SemEval2019Task6}
    \tabularnewline \bottomrule
  \end{tabularx}
\end{table}
