\section{Background}

\subsection{Previous Work on Mental Health of Soldiers}

To make the medical diagnosis for patients, psychologists often use the linguistic content and expression of patients to judge their emotional changes and mental state according to previous research in psychology and linguistic. The clinical diagnosis efficiency has been greatly improved because of the progress of science and technology, especially in computational linguistics. In addition, the widespread of social media such as Facebook, Twitter and Instagram has provided mental researchers with a large scale of data. Therefore, they can easily use the collected dataset and machine learning techniques for sentiment analysis.

Linguistic contents that users post on social media have been proved to be the basis for evaluating a person's mental state \citep{becauseIwastoldsomuch} \citep{GUNTUKU201743}. However, the majority of research targets are civilians. In this paper, veterans and civilians will be regarded as research targets. Westgate in \citep{doi:10.1176/appi.ps.201400283} has come up with a method about the evaluation of Veterans’ Suicide Risk. However, this paper will concentrate on analyzing the impact of the war on veterans' mental state by comparing the tweets posted by soldiers with the tweets generated by civilians. In addition, comprehensive sentiment analysis of veterans will be summarized.

\subsection{Sentiment and Emotion Analysis on Social Media}

Sentiment analysis has been applied in various fields such as the research about consumer behaviour was in the field of product marketing and the analysis of voter bias. The advances in Natural Language Processing and linguistic research have led to the development of different methods of sentiment analysis.

Nowadays, people tend to use social media such as Twitter to post tweets and express their opinions and emotions. Most of the tweets generated from Twitter accounts are public by default and easy to obtain. Also, the tweets are short(limited to 140 characters) and often appear with spelling mistakes and slangs. A tweet often comes with other features like spreading tweets(retweet) from other accounts. The factors mentioned above make Twitter a viable source to carry out sentiment analysis.

Generally speaking, the common approaches for sentiment analysis consist of two parts, including the machine learning techniques based and the lexicon-based. Analyzing users' social activities and calculating linguistic features of user-generated texts are the core of the machine learning algorithm. Compared with machine learning-based method, the lexicon based method is more direct and straight forward. In sentiment analysis, the typical task is finding the polarities of the given texts. The text content is either positive, negative or neutral. Lexicon based method could recognize and analyze the words about sentiment and other emoticons and hashtags which are associated with the sentiment.

Therefore, the sentiment lexicons are adopted for matching the words from tweets, thus analyzing and determining the polarities of the corpus.

\cite{azizan2019lexicon} performed sentiment analysis on Twitter data about movie review tweets using R and lexicon-based method. They found that the lexicon-based method is more effective than the machine learning based method under the same calculation cost. \cite{8073512} used a dictionary-based method and analyzed the results at aspect level and document level to predict the public’s sentiment using tweets about product review.

SentiWordNet and SenticNet, as open lexicons resources, have been developed in recent years. Sentiwordnet is a lexical resource which scores a text on three premises object, positivity, negativity and objectivity. It is an open-source software which is free to use and helps in extracting the sentiment of the text. Due to the high accuracy, the SentiWordNet 3.0 \citep{baccianella2010sentiwordnet} will be used as lexicons in this paper.

\cite{montejo2012random} have defined a work that uses SentiWordNet on Twitter data to identify the polarity of the sentiment of the users. They extract weighted vector and use it in the SentiWordNet to determine the polarity making it an unsupervised solution. We will use SentiWordNet on tweets in order to find the differences between the tweets of a soldier and that of a normal user.
