\section{Background}

\subsection{Previous Work on Mental Health of Veterans}

In order to make a medical diagnosis for patients, psychologists often use the
linguistic content and expression of patients to judge their emotional changes
and mental state according to previous research in psychology and linguistic.
The clinical diagnosis efficiency has been greatly improved because of the
progress of science and technology, especially in computational linguistics.
In addition, the widespread of social media such as Facebook, Twitter and
Instagram, has provided mental researchers with a large scale of data.
Therefore, they could easily use the collected dataset and machine learning
techniques for sentiment analysis. Linguistic contents which users posted on
social media have been proved to be the basis for evaluating a person's mental
state \citep{becauseIwastoldsomuch} \citep{GUNTUKU201743}. However,
the majority of research targets are normal people.

In this paper, veterans will be regarded as research targets. Westgate in
\citep{doi:10.1176/appi.ps.201400283} has come up with a method about the evaluation
of veterans' suicide risk. This paper will concentrate on analyzing
the impact of the war on veterans' mental state through the Twitters posted by
themselves before and after the war instead of focusing on the prediction of
the suicide risk of veterans. In addition, the comparison with the twitters
released by ordinary users will be presented. Finally, comprehensive
sentiment analysis of veterans will be summarized.

\subsection{Available Database on Sentiment Analysis}

Sentiment analysis is a very useful method for analysing the sentiment of an
article, tweets or reviews. The advances in Natural Language processing and
linguistic research have led to the development of different methods of
sentiment analysis. The sentiment analysis is an integral part of our work and
it is the fulcrum on which our hypothesis hangs on. Determining not just the
sentiment of a text but even the topic \citep{10.1145/1645953.1646003} on which
it is written on is one of the interesting works in this field. Another work
makes use of the lexicons in the text and word dictionaries to extract the
sentiment behind it \citep{10.5555/2002472.2002491}. There is another work that
trains a model to learn not just the words but to also capture the sentiment
behind it \citep{taboada2011lexicon}.

In this work, we do sentiment analysis on tweets and we compare the tweets of
two types of users. We utilise a simple model which basically just classifies
the tweets into Happy and sad/depressing sentiments. This is sufficient
considering our work is to identify the difference in the mental states between
the soldiers who have served in wars and common people.

\subsection{Sentiment and Emotion Analysis on Social Media}

Emotion and sentiment are treated as different concepts in psychology. The
definition of \enquote{emotion} is a complex psychological state, which plays an
important role in operating motivators. For \enquote{sentiment}, it is created based on
emotion to refer to a mental attitude. A survey providing more information can
be referred to by \citep{yue2018survey}.

In sentiment analysis, the typical task is finding the polarities of the given
texts. The tests are probably positive, negative or neutral. The approach is
often counting the word using and produce scores due to the lexicons.

There are commonly two approaches - analysing users' social activities and calculate linguistic features of user-generated texts. The sentiment analysis mainly focuses on short texts(tweets) generated from Twitter accounts, since most of the data is public by default and easy to obtain online. Also, the tweets are short(limited to 280 characters) and often appears with spelling mistakes and slangs. A tweet often comes with other features like spreading tweets(retweet) from other accounts. These methods mentioned above make the analysis on tweets a paradigm to explore.

SentiWordNet makes use of Opinion Mining, which is understanding the opinion of text more than the topic \citep{esuli2006sentiwordnet}. Sentiwordnet is a lexical resource which scores a text on three premises object, positivity and negativity. Synsets are the building blocks of the Sentiwordnet, they form a wordnet and the wordnet is associated with the three scores. The three scores help determine how objective, positive and negative the text is. Sentiwordnet is an open-source software which is free to use and helps in extracting the sentiment of the text. The latest version of this is SentiWordNet 3.0 \citep{baccianella2010sentiwordnet} which is being used in this project. The latest version of Sentiwordnet uses an updated Wordnet compared to the older version. The algorithm used is updated to include random walk step to refine the scores. There is also considerable improvement in the accuracy of Sentiwordnet 3.0. It is used in numerous projects for the analysis of reviews and other related matters to understand whether the text is subjective or objective.

\cite{montejo2012random} has defined a work that uses SentiWordNet on Twitter data to identify the polarity of sentiment of the users. They extract weighted vector and use it in the SentiWordNet to determine the polarity making it an unsupervisedsolution. We are going to be using SentiWordNet on tweets in order to understand the differences between the tweets of a soldier and that of a normal user.
