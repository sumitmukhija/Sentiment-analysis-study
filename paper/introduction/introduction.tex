\section{Introduction}

Social media platforms and microblogging websites are some of the most popular online stages for people to express their views. Twitter, undeniably is one of the leading applications in this assortment. People use Twitter to post their real-time opinions in the form of tweets. These tweets can be analyzed and certain inferences can be extracted. These inferences can subsequently be used for academic and business purposes.

One of the primary reasons that make Twitter a feasible choice is the diverse nature of users. In this research, we intend to analyze and compare the tweets of war-veterans and the general public. We believe wars have an impact on soldiers' psychological and emotional states. We try to prove this hypothesis by comparing their tweets to the tweets posted by the civilians. We collect public data using Twitter API of both veterans/soldiers and civilians. The accounts selected for the research all meet the criteria of having a minimum of 50 tweets and do not belong to any organisation. The tweets collected are processed using SentiWordNet to recognize the polarity of the tweets, we also count the number of words, the number of cuss words used. The data extracted from the processing of veteran/soldier tweets are compared with those of the civilians.

The remainder of the paper is organized as follows. We examine on the literature related to the topic, with papers related to previous works on the mental health of veterans, available databases on sentiment analysis and previous works done on sentiment analysis on social media in Section 2. In Section 3, we introduce our dataset and the experiment done on the dataset, with the results we have. In Section 4, we take a deeper look into the results and present the discussion. In Section 5 and 6, we conclude and bring up future works needed for the topic.
